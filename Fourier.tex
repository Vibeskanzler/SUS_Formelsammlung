\section{Signaldarstellung im Frequenz- und Bildbereich}
\subsection{Fourierreihe periodischer Signale}
Die Überlagerung von Sinusschwingungen zu einem periodischen,
nichtsinusförmigen Signal nennt man harmonische Synthese.
\subsubsection{Reelle Fourierreihe}
\begin{mdframed}[style=exercise]
    \begin{itemize}
        \item mit $sin$ und $cos$:
            \[
                f(t) = a_0 \sum_{k=1}^\infty [a_k\cdot cos(k\omega_1 t) b_k\cdot sin(k\omega_1 t)]
            \]
        \item mit Amplitude und Phase:
            \begin{align*}
                f(t) &= A_0+\sum_{k=1}^\infty [A_k\cdot cos(k\omega_1 t + \varphi_k)]\\
                    &= A_0+\sum_{k=1}^\infty [A_k\cdot sin(k\omega_1 t + \varphi_k-\frac{\pi}{2})]
            \end{align*}

            \texttt{\footnotesize Koeffizienten }
            \[
                A_0 = \frac{1}{T}\int_{t_0}^{T+t_0} f(t)dt
            \]
            \begin{align*}
                a_k = \frac{2}{T}\int_{t_0}^{T+t_0} f(t)\cdot cos(k\omega_1 t) dt\\
                b_k = \frac{2}{T}\int_{t_0}^{T+t_0} f(t)\cdot sin(k\omega_1 t) dt
            \end{align*}
    \end{itemize}
\end{mdframed}

\subsubsection{Komplexe Fourierreihe}
\begin{mdframed}[style=exercise]
    \[
        f(t)=\sum_{k=-\infty}^{\infty} \underline{c}_k\cdot e^{j\omega_1 k t}
    \]
    \begin{align*}
        \underline{c}_k &= \frac{1}{T}\int_{t_0}^{T+t_0} f(t)\cdot e^{-j\omega_1 k t}dt
                        &= \frac{1}{2}\left( a_k-jb_k \right)
    \end{align*}
\end{mdframed}

\subsubsection{Komplex Reell umwandeln}
\begin{mdframed}[style=exercise]
    \begin{gather*}
    \texttt{\footnotesize Komplex $\rightarrow$ Reell:}\\
                a_0 = A_0 = \underline{c}_0\\
                \left.
                \begin{split}
                    a_k = 2\ \mathfrak{Re}\left\{ c_k \right\}= \left[ \underline{c}_k+\underline{c}_{-k} \right]\\
                    b_k = -2\ \mathfrak{Im}\left\{ \underline{c}_k \right\}= j\left[ \underline{c}_k-\underline{c}_{-k} \right]\\
                    A_k = 2|\underline{c}_k| \quad \beta_k = -\varphi_k
                \end{split}\right\} \quad k>0\\
    \texttt{\footnotesize Reell $\rightarrow$ Komplex:}\\
                \left.
                \begin{split}
                    \underline{c}_k = \frac{1}{2}\left( a_k-jb_k \right) = \frac{A_k}{2} e^{-j\beta_k}\\
                    \underline{c}_{-k} = \frac{1}{2}\left( a_k+jb_k \right) = \frac{A_k}{2} e^{j\beta_k}
                \end{split}\right\} \quad k>0\\
    \end{gather*}
\end{mdframed}

\subsubsection{Symmetrieeigenschaften}
\begin{itemize}
    \item Gerade Funktionen
        symmetrisch zur y-Achse\\
        alle $sin$-teile verschwinden
        - $A_0 = \frac{2}{T}\int^{\frac{T}{2}}_{0} y(t)dt$\\
        - $a_{k} = \frac{4}{T}\int^{\frac{T}{2}}_{0}y(t)\cdot cos(k\omega_1t)dt$\\
        - $b_k = 0$\\
    \item Ungerade Funktionen
        symmetrisch zum Ursprung\\
        alle $cos$-teile und Gleichanteil verschwinden

        - $A_0 = 0$\\
        - $a_k = 0$\\
        - $b_{k} = \frac{4}{T}\int^{\frac{T}{2}}_{0} y(t)\cdot sin(k\omega_1t)dt$\\
\end{itemize}
\subsubsection{Halbwellensymmetrie}
Halbwellensymmetrie gilt wenn:
\[
    y(t) = -y(t \pm T/2)
\]
Die Fourier-Reihe einer Zeitfunktion mit HWS enthält stets
nur Terme mit ungeraden Ordnungszahlen. $k=1,3,5,\dots,\infty$
\begin{mdframed}[style=exercise,frametitle=im Allgemeinen]
    \texttt{\footnotesize Koeffizienten}:\\
    \[
        A_0 = 0,\
        a_{2k} = 0,\
        b_{2k} = 0
    \]
        $$a_{2k-1} = \frac{4}{T}\int^{\frac{T}{2}}_{0}y(t)\cdot cos((2k-1)\omega_1t)dt$$
        $$b_{2k-1} = \frac{4}{T}\int^{\frac{T}{2}}_{0}y(t)\cdot sin((2k-1)\omega_1t)dt$$
\end{mdframed}
\begin{mdframed}[style=exercise,frametitle=gerade Halbwellensymmetrie]
    \[
        A_0 = 0,\
        b_k = 0,\
        a_{2k} = 0
    \]
    $$a_{2k-1} = \frac{8}{T}\int^{\frac{T}{4}}_{0}y(t)\cdot cos((2k-1)\omega_1t)dt$$
\end{mdframed}
\begin{mdframed}[style=exercise,frametitle=ungerade Halbwellensymmetrie]
    \[
        A_0 = 0,\
        a_k = 0,\
        b_{2k} = 0
    \]
    $$b_{2k-1} = \frac{8}{T}\int^{\frac{T}{4}}_{0}y(t)\cdot sin((2k-1)\omega_1t)dt$$
\end{mdframed}

\subsubsection{Verschiebungssatz}
Verschiebung im Zeitbereich entspricht eine Drehung den Komplexen Spektrum um
die Phase $\rightarrow\ -k\omega_1 t_v$
\begin{align*}
    f_v(t) = f(t-t_v) &= \sum_{k=-\infty}^{\infty} \underline{c}_k\cdot e^{j\omega_1 k (t-t_v)} \\
           &= \sum_{k=-\infty}^{\infty} \underbrace{\underline{c}_k\cdot e^{j\omega_1 k t_v}}_{\underline{c}_{k_v}} \cdot e^{\omega_1 k t}
\end{align*}

Ist tv < 0, wie im Beispiel oben, so werden die Phasenwinkel des Spektrums mit
zunehmender Frequenz größer.

\subsubsection{Fourierreihe und LTI-Systeme}

\[
    y(t) = \sum_{k=-\infty}^{\infty} \underbrace{\underline{H}(k\omega_1)\cdot\underline{c}_{xk}}_{\underline{c}_{yk}} \cdot e^{j\omega_1 k t}
\]

\subsection{Kenngrößen periodischer Signale}
\begin{itemize}
    \item Effektivwert
        \[
            U_{\mathit{eff}} = \sqrt{\frac{1}{T} \int_\tau^{\tau+T} u(t)^2 dt}
        \]
        mit der Fourierreihe:

        $$ U_{\mathit{eff}} = \sqrt{\sum_{k=-\infty}^{\infty} c_k^2} = \sqrt{\sum_{k=-\infty}^{\infty}U_{k,\mathit{eff}}^2} $$
        auch:
        $$ \sqrt{A_0^2 + \frac{1}{2} \sum_{k=1}^{\infty} A_k^2} $$
    \item Klirrfaktor(Oberschwingungsgehalt):\\
        Dient zur Quantifizierung einer nichtlinearen Verzerrung bzw. von der
        Sinusform eines Signals.
        \begin{align*}
            k &= \frac{\text{Effektivwert der Oberschwingungen}}{\text{Effektivwert des Wechselanteil}} \\
                &= \frac{\sqrt{U_\sim^2 - U_1^2}}{U_\sim} \leq 1
        \end{align*}
        Für Wechselgrößen lässt sich $k$ einfach mit \textbf{Grundschwingungsgehalt} $g$ ermitteln (\textit{gilt immer}):
        \[
            k = \sqrt{1-g^2} \leftrightarrow g = \frac{U_1}{U}
        \]

\end{itemize}
